%-----------------------------------------------------------------
% UNIOESTE - Ci\^ encia da Computa\c c\~ ao
% 4o. ano - Trabalho de Conclus\~ ao de Curso
% Profa. Teresinha Arnauts Hachisuca 
%-----------------------------------------------------------------
%DECLARA\c C\~ AO DO TIPO DE DOCUMENTO, TAMANHO DA FOLHA E FONTE
%-----------------------------------------------------------------
\documentclass[
	% -- op\c c\~ oes da classe memoir --
	12pt,				% tamanho da fonte
% 	twoside,			% para impress\~ ao em verso e anverso. Oposto a oneside
	a4paper,			% tamanho do papel. 
	% -- op\c c\~ oes da classe abntex2 --
	%chapter=TITLE,		% t\' itulos de cap\' itulos convertidos em letras mai\' usculas
	%section=TITLE,		% t\' itulos de se\c c\~ oes convertidos em letras mai\' usculas
	%subsection=TITLE,	% t\' itulos de subse\c c\~ oes convertidos em letras mai\' usculas
	%subsubsection=TITLE,% t\' itulos de subsubse\c c\~ oes convertidos em letras mai\' usculas
	% -- op\c c\~ oes do pacote babel --
	english,			% idioma adicional para hifeniza\c c\~ ao
	brazil,				% o \' ultimo idioma \' e o principal do documento
	]{article}
	
%-----------------------------------------------------------------
%DEFINI\c C\~ OES DOS PACOTES UTILIZADOS
%-----------------------------------------------------------------
\usepackage{cmap}				% Mapear caracteres especiais no PDF
\usepackage{lmodern}			% Usa a fonte Latin Modern			
\usepackage[T1]{fontenc}		% Selecao de codigos de fonte.
\usepackage[utf8]{inputenc}		% Codificacao do documento (convers\~ ao autom\' atica dos acentos)
\usepackage{lastpage}			% Usado pela Ficha catalogr\' afica
%\usepackage{indentfirst}		% Indenta o primeiro par\' agrafo de cada se\c c\~ ao.
\usepackage{color}				% Controle das cores
\usepackage{graphicx}			% Inclus\~ ao de gr\' aficos

\usepackage[brazil]{babel}      % Permite traduzir termos do LateX para portugu\^ es Brasil.
\usepackage{hyperref}           % Permite ativar hyperlinks
\usepackage{algorithm}          % pacote que de suporte a representa\c c\~ ao de algoritmos.

\usepackage{algorithmic}
\usepackage[pdftex]{geometry}
\usepackage{multirow}

% ---
% Pacotes de cita\c c\~ oes
% ---
\usepackage[brazilian,hyperpageref]{backref}	 % Paginas com as cita\c c\~ oes na bibl
\usepackage[alf]{abntex2cite}	% Cita\c c\~ oes padr\~ ao ABNT
%\citebrackets[]

%\usepackage[backend=bibtex]{biblatex}


% --- 
% CONFIGURA\c C\~ OES DE PACOTES
% --- 

% ---
% Configura\c c\~ oes do pacote backref
% Usado sem a op\c c\~ ao hyperpageref de backref
\renewcommand{\backrefpagesname}{Citado na(s) p\' agina(s):~}

% Texto padr\~ ao antes do n\' umero das p\' aginas
\renewcommand{\backref}{}

% Define os textos da cita\c c\~ ao
\renewcommand*{\backrefalt}[4]{
	\ifcase #1 %
		Nenhuma cita\c c\~ ao no texto.%
	\or
		Citado na p\' agina #2.%
	\else
		Citado #1 vezes nas p\' aginas #2.%
	\fi}%
% ---

%-----------------------------------------------------------------
%               INICIO DO DOCUMENTO - CAPA
%-----------------------------------------------------------------
\begin{document}

%\nobibliography*

\begin{center}

	\textsc{
		\large
			\\universidade estadual do oeste do paran\' a
			\\unioeste - campus de foz do igua\c cu
			\\centro de engenharias e ci\^ encias exatas
			\\curso de ci\^ encia da computa\c c\~ ao
			\\[1 cm]tcc - trabalho de conclus\~ ao de curso
	}
	\\
	[4 cm]
	\large Proposta de Trabalho de Conclus\~ ao de Curso
    \\
	\textbf{
	    \textsc{Minera\c c\~ ao de texto da literatura m\' edica sobre a COVID-19}
    }
	\\[5 cm]Jedson Gabriel Ferreira de Paula
    \\Orientador: Rômulo César Silva
    %\\Co-orientador(es): \{Nome dos Co-orientadores\}
    \\[2 cm]Foz do Igua\c cu, 22 de junho de 2020
    
    
\end{center}

\thispagestyle{empty}

%-----------------------------------------------------------------
%               IDENTIFICA\c C\~ AO DO PROJETO
%-----------------------------------------------------------------
\section{Identifica\c c\~ ao}
    
    \subsection{Ci\^ encia da Computa\c c\~ ao} 
    
    % Consultar o link {\bf http://www.cnpq.br/documents/10157/186158/ TabeladeAreasdoConhecimento.pdf} 
    % para verificar as \' areas e linhas de pesquisa existentes.
    
        \noindent Grande \' area: Ci\^ encia da Computa\c c\~ ao
        \\C\' odigo: 1.03.00.00-7 
    	\\[1 cm]Linha de Pesquisa: Matem\' atica da Computa\c c\~ ao
    	\\C\' odigo: 1.03.02.00-0
    	\\[1 cm]Especialidade: Modelos Anal\' iticos e de Simula{\c c}\~ ao
    	\\C\' odigo: 1.03.02.02-6
    	
    	
    \subsection{Palavras-chave}

        \begin{enumerate}
        	\item minera\c c\~ ao de conhecimento
        	\item covid-19
        	\item coronav\' irus
        \end{enumerate}

%-----------------------------------------------------------------
%               INTRODU\c C\~ AO E JUSTIFICATIVA
%-----------------------------------------------------------------
\section{Introdu\c c\~ ao e Justificativa}

% O texto ainda está como uma cópia total da proposta e desejo ir iterando sobre ela

Segundo informa\c c\~ oes do Minist\' erio da Sa\' ude \cite{brasil20online}, a COVID-19 \' e uma doen\c ca causada pelo coronav\' irus SARS-CoV-2, que apresenta um quadro cl\' inico que varia de infec\c c\~ oes assintom\' aticas a quadros respirat\' orios graves. De acordo com a Organiza\c c\~ ao Mundial de Sa\' ude (OMS), a maioria dos pacientes com COVID-19 (cerca de 80\%) podem ser assintom\' aticos e cerca de 20\% dos casos podem requerer atendimento hospitalar por apresentarem dificuldade respirat\' oria e desses casos aproximadamente 5\% podem necessitar de suporte para o tratamento de insufici\^ encia respirat\' oria (suporte ventilat\' orio). Nos tr\^ es primeiros meses de 2020, a pandemia desse coronav\' irus atigiu v\' arios pa\' ises, incluindo o Brasil. As maiores preocupa\c c\~ oes atuais dos governos de diferentes pa\' ises \' e com o colapso nos sistemas de sa\' udes devido à sobrecarga de pacientes necessitando interna\c c\~ ao, e os preju\' izos econômicos e sociais devido ao confinamento e restri\c c\~ ao de deslocamento dos cidad\~ aos.

A Minera\c c\~ ao de Dados tem sido aplicada nas mais diversas \' areas do conhecimento com o intuito de descobrir novas informa\c c\~ oes e subsidiar as tomadas de decis\~ oes \cite{bramer2007principles, aggarwal2015data}. Atualmente existem diversas ferramentas e frameworks desenvolvidos especialmente para a minera\c c\~ ao de dados (data mining) e aprendizado de m\' aquina (machine learning) tais como: Scikit-Learn, Weka, R, Python e Pandas \cite{geron19}. Uma das aplica\c c\~ oes de minera\c c\~ ao de dados \' e o Processamento de Linguagem Natural (do ingl\^ es NLP – Natural Language Processing) e a minera\c c\~ ao de texto com o objetivo de mapeamento, extra\c c\~ ao de informa\c c\~ ao, entendimento de linguagem humana (natural) \cite{dinov2018data}. Tamb\' em t\^ em sido desenvolvidas diversas ferramentas open-source voltadas especificamente para minera\c c\~ ao de textos tais como: Aika, Rapidminer Text Mining, Data Science Toolkit e KNIME \cite{kaur2016comparison}. A minera\c c\~ ao de texto examina grande volumes de texto n\~ ao estruturado (corpus), auxiliando a extra\c c\~ ao de novas informa\c c\~ oes, descoberta de contexto, identifica\c c\~ ao de motivos lingu\' isticos, ou transformar o texto em formato de dados estruturado para derivar dados quantitativos que possam ser analisados futuramente \cite{aggarwal2015data}.

Uma das principais justificativas a favor do uso de minera\c c\~ ao de textos \' e a sobrecarga de informa\c c\~ ao devido ao grande volume de textos. Pode-se elencar entre as dificuldades para os humanos relacionadas a essa condi\c c\~ ao: profissionais se manterem atualizados com toda a literatura existente, encontrar informa\c c\~ ao precisa e relevante, sintetizar informa\c c\~ ao de fontes diversas e descobrir novos conhecimentos.

O dataset CORD-19 representa a mais extensa cole\c c\~ ao de literatura sobre coronav\' irus, leg\' ivel por m\' aquina, dispon\' ivel para minera\c c\~ ao de dados at\' e o momento \cite{kaggle20online}. Isto representa uma oportunidade de aplicar abordagens de minera\c c\~ ao de texto e dados para encontrar respostas a perguntas e conectar informa\c c\~ oes sobre esse conte\' udo em apoio aos esfor\c cos cont\' inuos de resposta ao COVID-19. H\' a uma crescente urg\^ encia para essas abordagens devido ao r\' apido aumento da literatura sobre o coronav\' irus, dificultando o acompanhamento da comunidade m\' edica. O site \citeonline{kaggle20online}, que promove competi\c c\~ oes/desafios em aprendizado de m\' aquina, tem elencado algumas quest\~ oes cient\' ificas importantes para minera\c c\~ ao de texto no dataset CORD-19, extra\' idas de t\' opicos de pesquisa do SCIED (Standing Committee on Emerging Infectious Diseases and 21st Century Health Threats) do NASEM (National Academies of Sciences, Engineering, and Medicine) nos EUA e da Organiza\c c\~ ao Mundial da Sa\' ude (OMS) para COVID-19.

% Este parágrafo mostra o objetivo ent\~ ao n\~ ao sei se deve ficar aqui
%Este projeto prop\~oe desenvolver uma ferramenta de minera\c c\~ ao de textos e aprendizagem de m\' aquina voltada para aplica\c c\~ ao no dataset CORD-19 com o objetivo de fornecer uma aplica\c c\~ ao que facilita a cataloga\c c\~ ao de textos acad\^ emicos de interesse da comunidade m\' edica, de acordo com crit\' erios da OMS.

%-----------------------------------------------------------------
%                       OBJETIVO
%-----------------------------------------------------------------	
\section{Objetivos}

% N\~ ao sei se deve ser colocado algum texto aqui
%As atividades a serem desenvolvidas durante o per\' iodo s\~ ao:

%-------------------------------------------------------------------------------

\subsection{Objetivo Geral}

% TODO: Melhorar objetivo
	Este projeto prop\~oe desenvolver uma ferramenta de minera\c c\~ ao de textos e aprendizagem de m\' aquina voltada para aplica\c c\~ ao no dataset CORD-19 com o objetivo de fornecer uma aplica\c c\~ ao que facilita a pesquisa de textos acad\^ emicos de interesse da comunidade m\' edica, de acordo com crit\' erios da OMS.

%-------------------------------------------------------------------------------

\subsection{Objetivos Espec\' ificos}

% TODO: reescrever objetivos espec\' ificos
{\bf Dentre os principais objetivos espec\' ificos destacam-se: }
\begin{itemize}
\item Pesquisa e revis\~ ao bibliogr\' afica sobre t\' ecnicas gerais de minera\c c\~ ao de textos e aprendizagem de m\' aquina
\item Estudar as caracter\' isticas/informa\c c\~ oes gerais da base CORD-19
\item Pesquisa e sele\c c\~ ao de quest\~ oes cient\' ificas relevantes referentes à COVID19 e o SARS-CoV-2
\item Implementa\c c\~ ao de t\' ecnicas identificadas como adequadas à minera\c c\~ ao de
textos relacionadas às quest\~ oes selecionadas
\item Interpreta\c c\~ ao/avalia\c c\~ ao dos resultados obtidos
\item Constru{\c c}\~ ao da aplica{\c c}\~ ao que utiliza das informa{\c c}\~ oes processadas (mineradas) para pesquisa  e recomenda{\c c}\~ ao de artigos relacionados
\item Reda\c c\~ ao de artigo cient\' ifico para que o aluno possa desenvolver a
capacidade de escrever de maneira cient\' ifica sobre o pr\' oprio processo de
pesquisa e os resultados obtidos
\end{itemize}

%-------------------------------------------------------------------------------


    	
%-----------------------------------------------------------------
%           PLANO DE TRABALHO E CRONOGRAMA DE EXECU\c C\~ AO
%-----------------------------------------------------------------	
\section{Plano de Trabalho e Cronograma de Execu\c c\~ ao}
	
    As atividades a serem desenvolvidas s\~ ao:

% TODO: Reescrever atividades
    \begin{enumerate}

        \item Revis\~ ao bibliogr\' afica sobre t\' ecnicas gerais de minera\c c\~ ao de texto e aprendizagem de m\' aquina e respectivas ferramentas open source (Python, Java, Scikit-Learn, R, Rapidminer Text Mining, Data Science Toolkit e KNIME);\label{a1}
        
        \item Estudo das caracter\' isticas gerais da base CORD-19 e quest\~ oes cient\' ificas relevantes referentes à COVID-19 e o SARS-CoV-2; \label{a2}

        \item Implementa\c c\~ ao da ferramenta proposta para minera\c c\~ ao de texto na base CORD-19;   \label{a3}

        \item An\' alise dos resultados obtidos pelos algoritmos de minera\c c\~ ao de textos e aprendizado de m\' aquina usados.   \label{a4}

        \item Elaboração e apresentação no tutorial de TCC (Trabalho de conclusão de cuso);  \label{a5}

        \item Constru{\c c}\~ ao da aplica{\c c}\~ ao para pesquisa e recomenda{\c c}\~ ao de artigos relacionados;   \label{a6}

         \item Reda{\c c}\~ ao do artigo descrevendo o processo de pesquisa e do resultado obtido;   \label{a7}

         \item Reda\c c\~ ao da monografia para submiss\~ ao a banca;   \label{a8}
    
\end{enumerate}

    Na Tabela \ref{tabela:cronograma1} \' e apresentado o cronograma de atividades a ser seguido.
 
\begin{table}[ht]
    \scriptsize
    \centering
    \begin{tabular}{|l|c|c|c|c|c|c|c|}
        \hline &  \multicolumn{7}{|c|}
        {\textbf{Per\' iodo}} \\ \cline{2-8}
        \textbf{Atividades}     &Nov      &Dez      &Jan      &Fev      &Mar	&Abr      &Jun	\\ \hline
        \ref{a1} - Revis\~ ao Bibliogr\' afica	&$\bullet$&$\bullet$&$\bullet$&	&	&	&	\\ \hline
        \ref{a2} - Estudo da base CORD-19	&$\bullet$&$\bullet$&$\bullet$&	&	&	&	\\ \hline
        \ref{a3} - Implementa\c c\~ ao da ferramenta	&$\bullet$&$\bullet$&$\bullet$&$\bullet$&$\bullet$&	&	\\ \hline
        \ref{a4} - An\' alise dos resultados	&	&$\bullet$&$\bullet$&$\bullet$&$\bullet$&$\bullet$&	\\ \hline
        \ref{a5} - Elaboração e apresentação no tutorial de TCC	&	&	&	&$\bullet$&	&	&	\\ \hline
        \ref{a6} - Constru{\c c}\~ ao da aplica{\c c}\~ ao	&	&$\bullet$&$\bullet$&$\bullet$&$\bullet$&$\bullet$&	\\ \hline
        \ref{a7} -  Reda\c c\~ ao do artigo	&	&	&$\bullet$&$\bullet$&$\bullet$&$\bullet$&	\\ \hline
        \ref{a8} - Reda\c c\~ ao da monografia	&	&	&$\bullet$&$\bullet$&$\bullet$&$\bullet$&$\bullet$	\\ \hline
%        \ref{a7} - Apresenta{\c c}\~ ao	&	&	&	&$\bullet$&$\bullet$&	\\ \hline

   \end{tabular}
     \caption{Cronograma das Atividades}
    \label{tabela:cronograma1}
\end{table}
 
%-----------------------------------------------------------------
%                   MATERIAL E M\' ETODO
%----------------------------------------------------------------- 

\section{Material e M\' etodo}

Para desenvolvimento deste projeto ser\~ ao usados os materiais:

% Acredito que devo colocar minhas especifica{\c c}\~ oes aqui
\begin{itemize}
\item Notebook Acer, 8 GB de memória RAM, 1 TB de HDD, 256 GB de SSD, Intel i7 7ª gera{\c c}\~ ao, Placa de v\' ideo Geforce 940MX;
\item Linguagem de programa\c c\~ ao javascript, python e suas bibliotecas cient\' ificas (Numpy, Pandas, Scikit-Learn)
\item Ferramentas open source voltadas especificamente para minera\c c\~ ao de textos (Aika, Rapidminer Text Mining, Data Science Toolkit e KNIME)
\item Quanto à metodologia de desenvolvimento de software ser\' a adotado modelo iterativo e incremental \cite{pressman2016engenharia}.
\end{itemize}

   A pesquisa bibliográfica será desenvolvida com fundamento em \cite{lakatos2001metodologia} nas seguintes etapas:
   \begin{enumerate}
\item Levantamento bibliográfico preliminar; \label{b1}
\item Busca das fontes; \label{b2}
\item Leitura do material; \label{b3}
\item Redação do texto; \label{b4}
   \end{enumerate}
   
   Sendo \ref{b1} listado na síntese bibliográfica deste documento. \ref{b2} será feito e documentado posteriormente na redação do texto. Todas estas etapas incorporarão a atividade \ref{a1} citada no cronograma \ref{tabela:cronograma1}.

Os dados que serão utilizados durante o processo de mineração são os encontrados no dataset do Kaggle \cite{kaggle20online}, nele contém a literatura médica sobre a doença e vírus o qual a comunidade médica possui interesse de estudo nos dias atuais.

O público que este texto está focado é o de ciência da computação pois as técnicas serão discutidas em termos de eficiência algoritmica para eficácia de classificação, já que o objeto a ser manipulado exige entendimento especializado. A aplicação que será desenvolvida já é para o uso da comunidade médica sendo levado em consideração a facilitação de pesquisa e exploração dos tópicos incorporados pelos textos acadêmicos.

A manipulação do texto será feita da seguinte forma: seus abstracts incorporados ao corpo do texto, mantendo uma refêrencia da sua origem para uso posterior; serão limpos e organizados os registros; e serão aplicadas as ferramentas estudadas no período da elaboração da revisão.

O resultado será analisado em termos de coerência dos clusters a partir dos tópicos incorporados nos textos.

%-----------------------------------------------------------------
%               CRIT\' ERIOS DE AVALIA\c C\~ AO
%-----------------------------------------------------------------
\section{Crit\' erios de Avalia\c c\~ ao} 

	Depois de escolhidas as caracter\' isticas para cria{\c c}\~ ao da matriz de features e feita a redu{\c c}\~ ao de dimensionalidade para realiza{\c c}\~ ao do clustering, teremos um espa{\c c}o preenchido com um cont\' inuo das variáveis latentes. E a qualidade destes clusters, isto é, o qu\~ ao bem definidos seus per\' imetros s\~ ao pode ser medido com um algor\' itmo que define o coeficiente da silhueta.
    
    
    
%-----------------------------------------------------------------
%                       REFER\^ ENCIAS
%----------------------------------------------------------------- 



\renewcommand\refname{}
\section{Refer\^ encias}

%Referencias bibliogr\' aficas que foram utilizadas para desenvolver a proposta de TCC.
    \vspace{-4.3em}
    \bibliography{referencias}
    
%-----------------------------------------------------------------
%                   S\' iNTESE BIBLIOGR\' aFICA
%----------------------------------------------------------------- 
\section{S\' intese Bibliogr\' afica}

%Referencias bibliogr\' aficas que se pretende utilizar para o desenvolvimento do trabalho.
    \vspace{-3.5em}
    %@book{alpaydin2020introduction,
%  title={Introduction to machine learning},
%  author={Alpaydin, Ethem},
%  year={2020},
%  publisher={MIT press}
%}

%@incollection{igual2017introduction,
%  title={Introduction to Data Science},
%  author={Igual, Laura and Segu{\'\i}, Santi},
%  booktitle={Introduction to Data Science},
%  pages={1--4},
%  year={2017},
%  publisher={Springer}
%}

%@book{skiena2017data,
%  title={The data science design manual},
%  author={Skiena, Steven S},
%  year={2017},
%  publisher={Springer}
%}

\begin{thebibliography}{}

%\bibitem[BINACIONAL I. 2015]{ib2014}
%\abntrefinfo{ib2014}{BINACIONAL}{2015}
%{BINACIONAL, I. \emph{Itaipu Binacional}.
%Dispon{\'i}vel em: <http://www.itaipu.gov. br>.
%Acessado em 10/02/2015. }

\bibitem[Igual, L. and Segu{\'\i}, S. 2017]{igual2017introduction}
\abntrefinfo{igual2017introduction}{Igual et al}{2017}
{IGUAL, L.; SEGU{\' I}, S. \emph{Introduction to Data Science}. Su{\' i}{\c c}a: Springer. 2017}

\bibitem[SKIENA, S. S. 2017]{skiena2017data}
\abntrefinfo{skiena2017data}{Skiena, Steven S}{2017}
{SKIENA, S S. \emph{The data science design manual}. Su{\' i}{\c c}a: Springer. 2017}

\bibitem[Alpaydin, E. 2020]{alpaydin2020introduction}
\abntrefinfo{alpaydin2020introduction}{Alpaydin}{2020}
{ALPAYDIN, E. \emph{Introduction to machine learning}.
2. ed. London, England: MIT Press. 2020}


\end{thebibliography}
 
\end{document}
